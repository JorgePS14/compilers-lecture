\documentclass[a4paper]{article}

\usepackage[english]{babel} \usepackage{hyperref} \usepackage{float}
\usepackage[utf8]{inputenc} \usepackage{amsmath} \usepackage{graphicx}
\usepackage[colorinlistoftodos]{todonotes} \usepackage{tikz}
\usepackage{pdfpages} \usepackage{listings}

\usetikzlibrary{arrows,positioning,shapes.geometric}

\title{Analysis of GCC's optimization feature}


\author{\normalsize Author: Jorge Andrés Pietra Santa Ochoa \\\normalsize}

\date{\color{black}October 27th, 2019}

\begin{document} \maketitle

\section{Abstract}

When we refer to Compiler Optimization, we refer to the process through which a compiler 
tries to optimize of maximize some of the aspects of a computer program. Commonly, the goal is 
to minimize a program's execution time, memory requirements and even the power consumption of said 
computer program. This is all done throught the analysis and identification of a programming language's 
patterns in order to create a binary executable that is easier for the CPU to read, interpret and process.
 

\section{Introduction}

More often than not, programming language engineers make performance one of the most important 
aspects to take into account when creating and designing a compiler so that the programs created by 
the language are as fast as they can be. One of the most common ways to know just how fast the compiler 
is is throguh the process of "benchmarking", which is basically comparing the results of one product, 
service or process (in this case we're benchmarking a process) using some key "comparators" that will 
make it evident for whoever is analyzing the results which one is better. In this specific case of analysis, 
we will be comparing the execution times of the binaries created by GCC when no optimizations are done, 
compared to the time taken by the file created when using optimization level 3.


\section{Objective}

In this document, we will analyze the assembly code generated by GCC with no optimizations and the code 
generated when we use the optimization parameter on level 3. The results might not be drastically different, 
but we should be able to observe at least some difference between the two.

\section{Development}

To be as transparent as possible, here are the specs of the machine that will be used to run these tests: 

\begin{table}[h]
    \centering \begin{tabular}{|l|l|}
        \hline Computer Model  &Macbook Pro 15 Mid-2018   \\
        \hline Operating System &macOS 10.15 Catalina   \\ 
        \hline Procesor Number &8750H   \\ 
        \hline \#Cores         &6       \\ 
        \hline \#Threads       &12      \\ 
        \hline Base Speed      &2.20 GHz \\
        \hline Turbo Speed     &4.10 GHz \\
        \hline \end{tabular} 
        \caption{Machine specs} 
    \label{table:Machine specs} 
\end{table}

The machine we're using for these tests is what the industry considers "high-end", so the time it will take for 
our computer to execute the files could be considerably faster than the average machine, so it might be a good 
idea to run these tests on your computer to observe the difference it might make on your specific setup.

The code we will be analyzing in this document is fairly simple, but it should give us an insigth of how much 
of a difference optimization can really make. Given the following C function: 

\begin{lstlisting} 
int proc(int a[]) {
	int sum = 0, i;
	for (i = 0; i < 1000000; i++)
		sum += a[i];
	return sum;
}
\end{lstlisting}

We will initialize a one-million-element array for the proc function to go through. We will later compile and execute 
the files generated by GCC.

First, let us take a look to both the unoptimized and optimized assembly codes generated by GCC, and see what the differences 
are:

\section{Unoptimized Assembly Code:}

\begin{lstlisting}
    ./nonOpt:	file format Mach-O 64-bit x86-64

    Disassembly of section __TEXT,__text:
    __text:
    100000ed0:	55 	pushq	%rbp
    100000ed1:	48 89 e5 	movq	%rsp, %rbp
    100000ed4:	48 89 7d f8 	movq	%rdi, -8(%rbp)
    100000ed8:	c7 45 f4 00 00 00 00 	movl	$0, -12(%rbp)
    100000edf:	c7 45 f0 00 00 00 00 	movl	$0, -16(%rbp)
    100000ee6:	81 7d f0 40 42 0f 00 	cmpl	$1000000, -16(%rbp)
    100000eed:	0f 8d 1f 00 00 00 	jge	31 <_proc+0x42>
    100000ef3:	48 8b 45 f8 	movq	-8(%rbp), %rax
    100000ef7:	48 63 4d f0 	movslq	-16(%rbp), %rcx
    100000efb:	8b 14 88 	movl	(%rax,%rcx,4), %edx
    100000efe:	03 55 f4 	addl	-12(%rbp), %edx
    100000f01:	89 55 f4 	movl	%edx, -12(%rbp)
    100000f04:	8b 45 f0 	movl	-16(%rbp), %eax
    100000f07:	83 c0 01 	addl	$1, %eax
    100000f0a:	89 45 f0 	movl	%eax, -16(%rbp)
    100000f0d:	e9 d4 ff ff ff 	jmp	-44 <_proc+0x16>
    100000f12:	8b 45 f4 	movl	-12(%rbp), %eax
    100000f15:	5d 	popq	%rbp
    100000f16:	c3 	retq
    100000f17:	66 0f 1f 84 00 00 00 00 00 	nopw	(%rax,%rax)
    100000f20:	55 	pushq	%rbp
    100000f21:	48 89 e5 	movq	%rsp, %rbp
    100000f24:	48 83 ec 20 	subq	$32, %rsp
    100000f28:	c7 45 fc 00 00 00 00 	movl	$0, -4(%rbp)
    100000f2f:	bf 00 09 3d 00 	movl	$4000000, %edi
    100000f34:	e8 2b 00 00 00 	callq	43 <dyld_stub_binder+0x100000f64>
    100000f39:	48 89 45 f0 	movq	%rax, -16(%rbp)
    100000f3d:	48 8b 7d f0 	movq	-16(%rbp), %rdi
    100000f41:	e8 8a ff ff ff 	callq	-118 <_proc>
    100000f46:	48 8d 3d 47 00 00 00 	leaq	71(%rip), %rdi
    100000f4d:	89 c6 	movl	%eax, %esi
    100000f4f:	b0 00 	movb	$0, %al
    100000f51:	e8 14 00 00 00 	callq	20 <dyld_stub_binder+0x100000f6a>
    100000f56:	31 f6 	xorl	%esi, %esi
    100000f58:	89 45 ec 	movl	%eax, -20(%rbp)
    100000f5b:	89 f0 	movl	%esi, %eax
    100000f5d:	48 83 c4 20 	addq	$32, %rsp
    100000f61:	5d 	popq	%rbp
    100000f62:	c3 	retq
    
    _proc:
    100000ed0:	55 	pushq	%rbp
    100000ed1:	48 89 e5 	movq	%rsp, %rbp
    100000ed4:	48 89 7d f8 	movq	%rdi, -8(%rbp)
    100000ed8:	c7 45 f4 00 00 00 00 	movl	$0, -12(%rbp)
    100000edf:	c7 45 f0 00 00 00 00 	movl	$0, -16(%rbp)
    100000ee6:	81 7d f0 40 42 0f 00 	cmpl	$1000000, -16(%rbp)
    100000eed:	0f 8d 1f 00 00 00 	jge	31 <_proc+0x42>
    100000ef3:	48 8b 45 f8 	movq	-8(%rbp), %rax
    100000ef7:	48 63 4d f0 	movslq	-16(%rbp), %rcx
    100000efb:	8b 14 88 	movl	(%rax,%rcx,4), %edx
    100000efe:	03 55 f4 	addl	-12(%rbp), %edx
    100000f01:	89 55 f4 	movl	%edx, -12(%rbp)
    100000f04:	8b 45 f0 	movl	-16(%rbp), %eax
    100000f07:	83 c0 01 	addl	$1, %eax
    100000f0a:	89 45 f0 	movl	%eax, -16(%rbp)
    100000f0d:	e9 d4 ff ff ff 	jmp	-44 <_proc+0x16>
    100000f12:	8b 45 f4 	movl	-12(%rbp), %eax
    100000f15:	5d 	popq	%rbp
    100000f16:	c3 	retq
    100000f17:	66 0f 1f 84 00 00 00 00 00 	nopw	(%rax,%rax)
    
    _main:
    100000f20:	55 	pushq	%rbp
    100000f21:	48 89 e5 	movq	%rsp, %rbp
    100000f24:	48 83 ec 20 	subq	$32, %rsp
    100000f28:	c7 45 fc 00 00 00 00 	movl	$0, -4(%rbp)
    100000f2f:	bf 00 09 3d 00 	movl	$4000000, %edi
    100000f34:	e8 2b 00 00 00 	callq	43 <dyld_stub_binder+0x100000f64>
    100000f39:	48 89 45 f0 	movq	%rax, -16(%rbp)
    100000f3d:	48 8b 7d f0 	movq	-16(%rbp), %rdi
    100000f41:	e8 8a ff ff ff 	callq	-118 <_proc>
    100000f46:	48 8d 3d 47 00 00 00 	leaq	71(%rip), %rdi
    100000f4d:	89 c6 	movl	%eax, %esi
    100000f4f:	b0 00 	movb	$0, %al
    100000f51:	e8 14 00 00 00 	callq	20 <dyld_stub_binder+0x100000f6a>
    100000f56:	31 f6 	xorl	%esi, %esi
    100000f58:	89 45 ec 	movl	%eax, -20(%rbp)
    100000f5b:	89 f0 	movl	%esi, %eax
    100000f5d:	48 83 c4 20 	addq	$32, %rsp
    100000f61:	5d 	popq	%rbp
    100000f62:	c3 	retq
    Disassembly of section __TEXT,__stubs:
    __stubs:
    100000f64:	ff 25 96 10 00 00 	jmpq	*4246(%rip)
    100000f6a:	ff 25 98 10 00 00 	jmpq	*4248(%rip)
    Disassembly of section __TEXT,__stub_helper:
    __stub_helper:
    100000f70:	4c 8d 1d 99 10 00 00 	leaq	4249(%rip), %r11
    100000f77:	41 53 	pushq	%r11
    100000f79:	ff 25 81 00 00 00 	jmpq	*129(%rip)
    100000f7f:	90 	nop
    100000f80:	68 00 00 00 00 	pushq	$0
    100000f85:	e9 e6 ff ff ff 	jmp	-26 <__stub_helper>
    100000f8a:	68 0e 00 00 00 	pushq	$14
    100000f8f:	e9 dc ff ff ff 	jmp	-36 <__stub_helper>
\end{lstlisting}

\section{Optimized Assembly Code:}

\begin{lstlisting}
    ./opt3:	file format Mach-O 64-bit x86-64

    Disassembly of section __TEXT,__text:
    __text:
    100000de0:	55 	pushq	%rbp
    100000de1:	48 89 e5 	movq	%rsp, %rbp
    100000de4:	66 0f ef c0 	pxor	%xmm0, %xmm0
    100000de8:	b8 24 00 00 00 	movl	$36, %eax
    100000ded:	66 0f ef c9 	pxor	%xmm1, %xmm1
    100000df1:	66 2e 0f 1f 84 00 00 00 00 00 	nopw	%cs:(%rax,%rax)
    100000dfb:	0f 1f 44 00 00 	nopl	(%rax,%rax)
    100000e00:	f3 0f 6f 94 87 70 ff ff ff 	movdqu	-144(%rdi,%rax,4), %xmm2
    100000e09:	66 0f fe d0 	paddd	%xmm0, %xmm2
    100000e0d:	f3 0f 6f 44 87 80 	movdqu	-128(%rdi,%rax,4), %xmm0
    100000e13:	66 0f fe c1 	paddd	%xmm1, %xmm0
    100000e17:	f3 0f 6f 4c 87 90 	movdqu	-112(%rdi,%rax,4), %xmm1
    100000e1d:	f3 0f 6f 5c 87 a0 	movdqu	-96(%rdi,%rax,4), %xmm3
    100000e23:	f3 0f 6f 64 87 b0 	movdqu	-80(%rdi,%rax,4), %xmm4
    100000e29:	66 0f fe e1 	paddd	%xmm1, %xmm4
    100000e2d:	66 0f fe e2 	paddd	%xmm2, %xmm4
    100000e31:	f3 0f 6f 54 87 c0 	movdqu	-64(%rdi,%rax,4), %xmm2
    100000e37:	66 0f fe d3 	paddd	%xmm3, %xmm2
    100000e3b:	66 0f fe d0 	paddd	%xmm0, %xmm2
    100000e3f:	f3 0f 6f 4c 87 d0 	movdqu	-48(%rdi,%rax,4), %xmm1
    100000e45:	f3 0f 6f 5c 87 e0 	movdqu	-32(%rdi,%rax,4), %xmm3
    100000e4b:	f3 0f 6f 44 87 f0 	movdqu	-16(%rdi,%rax,4), %xmm0
    100000e51:	66 0f fe c1 	paddd	%xmm1, %xmm0
    100000e55:	66 0f fe c4 	paddd	%xmm4, %xmm0
    100000e59:	f3 0f 6f 0c 87 	movdqu	(%rdi,%rax,4), %xmm1
    100000e5e:	66 0f fe cb 	paddd	%xmm3, %xmm1
    100000e62:	66 0f fe ca 	paddd	%xmm2, %xmm1
    100000e66:	48 83 c0 28 	addq	$40, %rax
    100000e6a:	48 3d 64 42 0f 00 	cmpq	$1000036, %rax
    100000e70:	75 8e 	jne	-114 <_proc+0x20>
    100000e72:	66 0f fe c8 	paddd	%xmm0, %xmm1
    100000e76:	66 0f 70 c1 4e 	pshufd	$78, %xmm1, %xmm0
    100000e7b:	66 0f fe c1 	paddd	%xmm1, %xmm0
    100000e7f:	66 0f 70 c8 e5 	pshufd	$229, %xmm0, %xmm1
    100000e84:	66 0f fe c8 	paddd	%xmm0, %xmm1
    100000e88:	66 0f 7e c8 	movd	%xmm1, %eax
    100000e8c:	5d 	popq	%rbp
    100000e8d:	c3 	retq
    100000e8e:	66 90 	nop
    100000e90:	55 	pushq	%rbp
    100000e91:	48 89 e5 	movq	%rsp, %rbp
    100000e94:	bf 00 09 3d 00 	movl	$4000000, %edi
    100000e99:	e8 c4 00 00 00 	callq	196 <dyld_stub_binder+0x100000f62>
    100000e9e:	66 0f ef c0 	pxor	%xmm0, %xmm0
    100000ea2:	b9 1d 00 00 00 	movl	$29, %ecx
    100000ea7:	66 0f ef c9 	pxor	%xmm1, %xmm1
    100000eab:	eb 1a 	jmp	26 <_main+0x37>
    100000ead:	0f 1f 00 	nopl	(%rax)
    100000eb0:	f3 0f 6f 44 88 f0 	movdqu	-16(%rax,%rcx,4), %xmm0
    100000eb6:	f3 0f 6f 0c 88 	movdqu	(%rax,%rcx,4), %xmm1
    100000ebb:	66 0f fe c2 	paddd	%xmm2, %xmm0
    100000ebf:	66 0f fe cb 	paddd	%xmm3, %xmm1
    100000ec3:	48 83 c1 20 	addq	$32, %rcx
    100000ec7:	f3 0f 6f 5c 88 90 	movdqu	-112(%rax,%rcx,4), %xmm3
    100000ecd:	66 0f fe d8 	paddd	%xmm0, %xmm3
    100000ed1:	f3 0f 6f 44 88 a0 	movdqu	-96(%rax,%rcx,4), %xmm0
    100000ed7:	66 0f fe c1 	paddd	%xmm1, %xmm0
    100000edb:	f3 0f 6f 4c 88 b0 	movdqu	-80(%rax,%rcx,4), %xmm1
    100000ee1:	f3 0f 6f 64 88 c0 	movdqu	-64(%rax,%rcx,4), %xmm4
    100000ee7:	f3 0f 6f 54 88 d0 	movdqu	-48(%rax,%rcx,4), %xmm2
    100000eed:	66 0f fe d1 	paddd	%xmm1, %xmm2
    100000ef1:	66 0f fe d3 	paddd	%xmm3, %xmm2
    100000ef5:	f3 0f 6f 5c 88 e0 	movdqu	-32(%rax,%rcx,4), %xmm3
    100000efb:	66 0f fe dc 	paddd	%xmm4, %xmm3
    100000eff:	66 0f fe d8 	paddd	%xmm0, %xmm3
    100000f03:	48 81 f9 3d 42 0f 00 	cmpq	$999997, %rcx
    100000f0a:	75 a4 	jne	-92 <_main+0x20>
    100000f0c:	66 0f fe da 	paddd	%xmm2, %xmm3
    100000f10:	66 0f 70 c3 4e 	pshufd	$78, %xmm3, %xmm0
    100000f15:	66 0f fe c3 	paddd	%xmm3, %xmm0
    100000f19:	66 0f 70 c8 e5 	pshufd	$229, %xmm0, %xmm1
    100000f1e:	66 0f fe c8 	paddd	%xmm0, %xmm1
    100000f22:	66 0f 7e ce 	movd	%xmm1, %esi
    100000f26:	03 b0 e4 08 3d 00 	addl	3999972(%rax), %esi
    100000f2c:	03 b0 e8 08 3d 00 	addl	3999976(%rax), %esi
    100000f32:	03 b0 ec 08 3d 00 	addl	3999980(%rax), %esi
    100000f38:	03 b0 f0 08 3d 00 	addl	3999984(%rax), %esi
    100000f3e:	03 b0 f4 08 3d 00 	addl	3999988(%rax), %esi
    100000f44:	03 b0 f8 08 3d 00 	addl	3999992(%rax), %esi
    100000f4a:	03 b0 fc 08 3d 00 	addl	3999996(%rax), %esi
    100000f50:	48 8d 3d 3d 00 00 00 	leaq	61(%rip), %rdi
    100000f57:	31 c0 	xorl	%eax, %eax
    100000f59:	e8 0a 00 00 00 	callq	10 <dyld_stub_binder+0x100000f68>
    100000f5e:	31 c0 	xorl	%eax, %eax
    100000f60:	5d 	popq	%rbp
    100000f61:	c3 	retq

    _proc:
    100000de0:	55 	pushq	%rbp
    100000de1:	48 89 e5 	movq	%rsp, %rbp
    100000de4:	66 0f ef c0 	pxor	%xmm0, %xmm0
    100000de8:	b8 24 00 00 00 	movl	$36, %eax
    100000ded:	66 0f ef c9 	pxor	%xmm1, %xmm1
    100000df1:	66 2e 0f 1f 84 00 00 00 00 00 	nopw	%cs:(%rax,%rax)
    100000dfb:	0f 1f 44 00 00 	nopl	(%rax,%rax)
    100000e00:	f3 0f 6f 94 87 70 ff ff ff 	movdqu	-144(%rdi,%rax,4), %xmm2
    100000e09:	66 0f fe d0 	paddd	%xmm0, %xmm2
    100000e0d:	f3 0f 6f 44 87 80 	movdqu	-128(%rdi,%rax,4), %xmm0
    100000e13:	66 0f fe c1 	paddd	%xmm1, %xmm0
    100000e17:	f3 0f 6f 4c 87 90 	movdqu	-112(%rdi,%rax,4), %xmm1
    100000e1d:	f3 0f 6f 5c 87 a0 	movdqu	-96(%rdi,%rax,4), %xmm3
    100000e23:	f3 0f 6f 64 87 b0 	movdqu	-80(%rdi,%rax,4), %xmm4
    100000e29:	66 0f fe e1 	paddd	%xmm1, %xmm4
    100000e2d:	66 0f fe e2 	paddd	%xmm2, %xmm4
    100000e31:	f3 0f 6f 54 87 c0 	movdqu	-64(%rdi,%rax,4), %xmm2
    100000e37:	66 0f fe d3 	paddd	%xmm3, %xmm2
    100000e3b:	66 0f fe d0 	paddd	%xmm0, %xmm2
    100000e3f:	f3 0f 6f 4c 87 d0 	movdqu	-48(%rdi,%rax,4), %xmm1
    100000e45:	f3 0f 6f 5c 87 e0 	movdqu	-32(%rdi,%rax,4), %xmm3
    100000e4b:	f3 0f 6f 44 87 f0 	movdqu	-16(%rdi,%rax,4), %xmm0
    100000e51:	66 0f fe c1 	paddd	%xmm1, %xmm0
    100000e55:	66 0f fe c4 	paddd	%xmm4, %xmm0
    100000e59:	f3 0f 6f 0c 87 	movdqu	(%rdi,%rax,4), %xmm1
    100000e5e:	66 0f fe cb 	paddd	%xmm3, %xmm1
    100000e62:	66 0f fe ca 	paddd	%xmm2, %xmm1
    100000e66:	48 83 c0 28 	addq	$40, %rax
    100000e6a:	48 3d 64 42 0f 00 	cmpq	$1000036, %rax
    100000e70:	75 8e 	jne	-114 <_proc+0x20>
    100000e72:	66 0f fe c8 	paddd	%xmm0, %xmm1
    100000e76:	66 0f 70 c1 4e 	pshufd	$78, %xmm1, %xmm0
    100000e7b:	66 0f fe c1 	paddd	%xmm1, %xmm0
    100000e7f:	66 0f 70 c8 e5 	pshufd	$229, %xmm0, %xmm1
    100000e84:	66 0f fe c8 	paddd	%xmm0, %xmm1
    100000e88:	66 0f 7e c8 	movd	%xmm1, %eax
    100000e8c:	5d 	popq	%rbp
    100000e8d:	c3 	retq
    100000e8e:	66 90 	nop

    _main:
    100000e90:	55 	pushq	%rbp
    100000e91:	48 89 e5 	movq	%rsp, %rbp
    100000e94:	bf 00 09 3d 00 	movl	$4000000, %edi
    100000e99:	e8 c4 00 00 00 	callq	196 <dyld_stub_binder+0x100000f62>
    100000e9e:	66 0f ef c0 	pxor	%xmm0, %xmm0
    100000ea2:	b9 1d 00 00 00 	movl	$29, %ecx
    100000ea7:	66 0f ef c9 	pxor	%xmm1, %xmm1
    100000eab:	eb 1a 	jmp	26 <_main+0x37>
    100000ead:	0f 1f 00 	nopl	(%rax)
    100000eb0:	f3 0f 6f 44 88 f0 	movdqu	-16(%rax,%rcx,4), %xmm0
    100000eb6:	f3 0f 6f 0c 88 	movdqu	(%rax,%rcx,4), %xmm1
    100000ebb:	66 0f fe c2 	paddd	%xmm2, %xmm0
    100000ebf:	66 0f fe cb 	paddd	%xmm3, %xmm1
    100000ec3:	48 83 c1 20 	addq	$32, %rcx
    100000ec7:	f3 0f 6f 5c 88 90 	movdqu	-112(%rax,%rcx,4), %xmm3
    100000ecd:	66 0f fe d8 	paddd	%xmm0, %xmm3
    100000ed1:	f3 0f 6f 44 88 a0 	movdqu	-96(%rax,%rcx,4), %xmm0
    100000ed7:	66 0f fe c1 	paddd	%xmm1, %xmm0
    100000edb:	f3 0f 6f 4c 88 b0 	movdqu	-80(%rax,%rcx,4), %xmm1
    100000ee1:	f3 0f 6f 64 88 c0 	movdqu	-64(%rax,%rcx,4), %xmm4
    100000ee7:	f3 0f 6f 54 88 d0 	movdqu	-48(%rax,%rcx,4), %xmm2
    100000eed:	66 0f fe d1 	paddd	%xmm1, %xmm2
    100000ef1:	66 0f fe d3 	paddd	%xmm3, %xmm2
    100000ef5:	f3 0f 6f 5c 88 e0 	movdqu	-32(%rax,%rcx,4), %xmm3
    100000efb:	66 0f fe dc 	paddd	%xmm4, %xmm3
    100000eff:	66 0f fe d8 	paddd	%xmm0, %xmm3
    100000f03:	48 81 f9 3d 42 0f 00 	cmpq	$999997, %rcx
    100000f0a:	75 a4 	jne	-92 <_main+0x20>
    100000f0c:	66 0f fe da 	paddd	%xmm2, %xmm3
    100000f10:	66 0f 70 c3 4e 	pshufd	$78, %xmm3, %xmm0
    100000f15:	66 0f fe c3 	paddd	%xmm3, %xmm0
    100000f19:	66 0f 70 c8 e5 	pshufd	$229, %xmm0, %xmm1
    100000f1e:	66 0f fe c8 	paddd	%xmm0, %xmm1
    100000f22:	66 0f 7e ce 	movd	%xmm1, %esi
    100000f26:	03 b0 e4 08 3d 00 	addl	3999972(%rax), %esi
    100000f2c:	03 b0 e8 08 3d 00 	addl	3999976(%rax), %esi
    100000f32:	03 b0 ec 08 3d 00 	addl	3999980(%rax), %esi
    100000f38:	03 b0 f0 08 3d 00 	addl	3999984(%rax), %esi
    100000f3e:	03 b0 f4 08 3d 00 	addl	3999988(%rax), %esi
    100000f44:	03 b0 f8 08 3d 00 	addl	3999992(%rax), %esi
    100000f4a:	03 b0 fc 08 3d 00 	addl	3999996(%rax), %esi
    100000f50:	48 8d 3d 3d 00 00 00 	leaq	61(%rip), %rdi
    100000f57:	31 c0 	xorl	%eax, %eax
    100000f59:	e8 0a 00 00 00 	callq	10 <dyld_stub_binder+0x100000f68>
    100000f5e:	31 c0 	xorl	%eax, %eax
    100000f60:	5d 	popq	%rbp
    100000f61:	c3 	retq
    Disassembly of section __TEXT,__stubs:
    __stubs:
    100000f62:	ff 25 98 10 00 00 	jmpq	*4248(%rip)
    100000f68:	ff 25 9a 10 00 00 	jmpq	*4250(%rip)
    Disassembly of section __TEXT,__stub_helper:
    __stub_helper:
    100000f70:	4c 8d 1d 99 10 00 00 	leaq	4249(%rip), %r11
    100000f77:	41 53 	pushq	%r11
    100000f79:	ff 25 81 00 00 00 	jmpq	*129(%rip)
    100000f7f:	90 	nop
    100000f80:	68 00 00 00 00 	pushq	$0
    100000f85:	e9 e6 ff ff ff 	jmp	-26 <__stub_helper>
    100000f8a:	68 0e 00 00 00 	pushq	$14
    100000f8f:	e9 dc ff ff ff 	jmp	-36 <__stub_helper>
\end{lstlisting}

\section{Test results}

Before conducting the tests, we first calculated the estimated run times using the following formula:

t=frac{I*CPI}{f}

Where t stands for time, I stands for the number of executed instructions, CPI is the average number of Clocks per Instruction, 
and f is our Clock Frequency.

For the CPIs, we weren't able to find a table that contained the specific CPIs for the Coffee Lake 14nm++ architecture from Intel,
but since Coffee Lake is built on the same 14nms process as Kaby Lake, and Kaby Lake is built on the same 14 nm process as Sylake,
we took the table for the enthusiast-grade Skylake-X architecture and figured they would perform similarly given the 14 nm process 
situation and performance gains reported by reviewers and benchmarks available online.

Having said that, using the CPIs from Skylake-X for our calculations, we obtained an estimated time of 1.2195 ms for the execution of 
our non-optimized file, whereas for the optimized file, we obtained 4.3488 ms. This is to be expected, since the optimization process opts 
for a different approach than usual: instead of repeating the same instruction several times, we instead go for more instructions but each 
one is repeated less times. The estimated latency is not necessarily a definitive equivalent of what is real-world performance, since the CPI 
is based on averages, and different instructions take different times to execute, so we corroborate our results with the time command provided 
by bash:

\begin{lstlisting}
time ./nonOpt
0.37 real         0.00 user         0.00 sys
\end{lstlisting} 

\begin{lstlisting}
time ./opt3
0.10 real         0.00 user         0.00 sys
\end{lstlisting} 

\begin{lstlisting}
time ./nonOpt
0.31 real         0.00 user         0.00 sys
\end{lstlisting} 

\begin{lstlisting}
time ./opt3
0.09 real         0.00 user         0.00 sys
\end{lstlisting}

\begin{lstlisting}
time ./nonOpt
0.08 real         0.00 user         0.00 sys
\end{lstlisting} 

\begin{lstlisting}
time ./opt3
0.07 real         0.00 user         0.00 sys
\end{lstlisting} 

\begin{lstlisting}
time ./nonOpt
0.07 real         0.00 user         0.00 sys
\end{lstlisting} 

\begin{lstlisting}
time ./opt3
0.06 real         0.00 user         0.00 sys
\end{lstlisting}

We ran each one four times: two with from a cold start (that is after leaving the computer idle for some time at below-base speeds 
--30 minutes was our standard), and two with an already warm CPU, that is, running at the full 4.1 GHz Turbo Frequency that our processor 
allows us to use.

\section{Conslusions}

We can observe from our results that even though our estimates were a bit off, perhaps because we were not using the official CPIs for the
Coffee Lake processors, but rather the CPI table for an enthusiast-grade lineup of CPUs, we can actually observe a difference between the optimized 
and unoptimized versions of our files.

From a cold start, the execution times differ vastly, with the unoptimized version taking anywhere from 0.30 to 0.40 seconds to execure, while the 
optimized version took only around 0.10 seconds. This is a performance gain of nearly 70\%. However, when running on a warm CPU, things weren't as drastic 
as with a cold CPU; the unoptimized program ran only about 0.01 seconds slower than the optimized version, whihc is still a performance gain, but maybe 
not as much as we would expect. However, this was using a fairly simple function with a quite simple task. This 0.01 second difference could add up and 
translate to potentially huge gains when dealing with huge programs.

\section{References: }

Fog, A. (2019). Lists of instruction latencies, throughputs and micro-operation breakdowns for Intel, AMD, and VIA CPUs. https://www.agner.org/optimize/instruction_tables.pdf